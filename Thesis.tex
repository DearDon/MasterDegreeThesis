
\documentclass[a4paper,12pt,single,pdftex]{scrartcl}
\usepackage{ngerman}
 \usepackage{color}  
 \usepackage{html}  
 \usepackage{times}  
 \usepackage{graphicx} 
 \usepackage{fancyheadings}  
 \usepackage{hyperref}  
 \setlength{\parindent}{0.6pt} 
 \setlength{\parskip}{0.6pt} 
 \title{Draft}
 

\begin{document} 
\maketitle
\newpage

\label{ID_492861276}\label{ID_1956039038}\section{引言}

\label{ID_1883815321}\subsection{研究意义}

\label{ID_84459052}\subsubsection{众所周知地震灾害是关系到国计民生的重大自然灾害,虽然极具破坏的大地震发生频率很低,但是一次地震所爆发的能量却是与吨级核爆相当\citep{Stein2003},其破坏性不言而喻。2008年5月12号的汶川地震是自唐山地震以来在我国发生的导致直接死亡人数最多,经济损失最大的重大地震。然而,另一方面,地震的高能量所激发的高穿透力的地震波却是地震学家研究地球结构的福音,是人类目前研究地球内部结构的最有力工具。所以,无论从人民生活安全,经济保障,还是从科学探索的角度看,地震学研究都是很有意义的。}

\label{ID_219947997}\subsubsection{地震学研究的主要内容是震源以及地下结构研究,震源机制是震源最基本的参数,其结果可进一步应用于理论震动图计算\citep{Wald2005}、海啸模拟\citep{Satake2007}、库仑应力转移估计\citep{King2007}、区域的应力分析和震源破裂过程反演中\citep{Kilb2001}。此外,利用已知震源机制计算得到面波震动图,用于在介质结构研究中代替到时或面波频散数据,以获得更多约束信息,直接拟合实际波形反演地下结构的方法也得到了广泛应用\citep{Nolet1990,Manaman2011,Friederich2003,Zielhuis1994,Cao2001,Lee1997}。因而在地震发生后,获得可靠的震源机制是有益且必要的。}

\label{ID_1577841117}\subsection{研究发展历程}

\label{ID_851024883}\subsubsection{}

\label{ID_1436540560}\subsubsection{用地震波形拟合反演震源机制时,由于待求参数少、解空间有限且波形与震源机制的关系是非线性的,所以该问题非常适合用非线性反演中的全局搜索算法。在实践中,得到广泛应用的CAP(Cut and Paste的简称)\citep{Zhao1994,Zhu1996,Tan2006}和CPS(Computer Programs in Seismology的简称)\citep{Herrmann1989}等波形拟合反演程序充分说明了全局搜索在震源机制反演问题中的有效性。}

\label{ID_1102145719}\subsection{研究现状及问题}

\label{ID_138447139}\subsubsection{Herrmann开发的CPS软件包中用于震源机制反演子程序的方法\citep{Herrmann1989}(为方便本文简称CPS方法)和CAP方法\citep{Zhao1994,Zhu1996,Tan2006}均为震源机制格点搜索方法,由于二者的实用性和开源性,被广泛应用于震源机制研究中。然而CAP和CPS方法加权的侧重点不同,前者考虑几何扩散导致波形振幅的衰减,给较低振幅的波形加大权重,以平衡不同振幅波形在反演中的影响力;后者则关注传播过程中信噪比的降低,赋予高信噪比数据较大权重。考虑到波形振幅及信噪比与震中距间的关系,Zhu等\citep{Zhu1996}在CAP中将权重设置为关于震中距递增的幂函数,而CPS方法中使用震中距的反比例函数作为权重值。幂函数和反比例函数的单调性恰巧相反,导致从数值上看,两种方法对同一套数据波形所定权重大小相互矛盾——CPS定权震中距较近台站权重较高,而CAP定权中则震中距较远台站相对权重较大。此外,通过实例计算及理论分析发现,实际观测波形的振幅及信噪比与震中距的关系较为复杂,难以用简单的初等函数进行描述,且函数中包含的参数赋值因人而异。因此无论利用幂函数或反比例函数估算的振幅和信噪比均较粗糙,无法准确体现数据的真实性和客观性。}

\label{ID_1797636015}\subsubsection{另一方面,随着CAP,CPS等用波形非线性反演震源机制的算法得到广泛应用\citep{Luo2015,D’Amico2014,赵博2013},其非线性反演中误差缺失的问题逐渐受到了关注。考虑到误差评价的重要性,国内外学者均对震源机制反演的误差估计问题进行了研究,Duputel等\citep{Duputel2012}从误差的来源入手,对震源机制进行误差评价,但其方法只适用于估计线性反演的震源机制误差。考虑到目前应用最广泛的全局搜索反演,本文旨在探究能应用于非性线反演算法的误差评估方法。对于非线性反演,最常见的方法是对目标函数的极值人为给定一个阈值,满足该条件的所有解构成误差范围内解集(简称为阈值法),该方法简洁直观,能快速发现不同模型参数的误差大小关系,但是阈值的给定有主观性,导致定量结果难以让人信服。郑建常等\citep{郑建常2015}借鉴Bootstrap(Efron1979)的思想,对数据集随机多次选取子集进行独立反演并对解集样本用一定方法分析,以评估其整体误差及解,为使样本能反映整体。但是统计分析不仅要求样本抽取的随机性,还对原始样本大小有一定要求,当可用的地震记录数量不是很大时基于重抽样的该方法便不适用了。}

\label{ID_616390154}\subsection{本文设定目标及方案}

\label{ID_1218200436}\subsubsection{针对以上分析,为了解决CPS和CAP反演定权方法出现的矛盾以及误差缺失问题,本文分别尝试进行权重优化以及误差评定。对于定权,结合CPS与CAP,综合考虑振幅衰减和信噪比差异影响,将二者统一计算权重,从而解决上述的权重数值矛盾问题。其次,计算时摈弃了用震中距简单函数估算振幅或信噪比方案,而是基于每道波形数据自身的信息,准确评估振幅和噪声。由于没有人工干预,在提高精确度的同时有效保证了客观性。}

\label{ID_1099326584}\subsubsection{为了解决反演震源机制时误差估计的问题,本文借鉴了\citet{Hardebeck2002}对P波初动极性反演方法改进的思路——首先估计观测数据的随机误差大小,据此模拟随机误差,并将其加入原始数据生成多组模拟数据集,最终反演得到一系列误差范围内的解集。该方法不仅估计了误差,且使得反演结果更稳定\citep{Hardebeck2002}。将该思想应用到波形反演震源机制的问题中,利用评估噪声随机模拟多套波形数据集,并用每套数据分别进行反演,得到包含多次反演结果的解集,并对解集统计分析得到解的误差(为方便叙述称其为模拟分析法)。本文方法与重采样类方法\citep{郑建常2015,Efron1979}不同的是每次反演的数据并非原始数据集的子集,而是与数据集等价的模拟数据集,保留了原数据集的样本大小,更重要的是每套数据均具有一致的数据分布结构。所以对观测数据的数量要求相对降低,能比重采样类方法更好应用于台站记录较稀少的地震事件。}

\label{ID_323720947}\subsubsection{为了验证本文提出权重优化方案和误差评价方法的有效性,基于CPS反演程序进行了一系列理论试验,分别检验权重优化效果和误差估计与理论预测是否吻合。对同一个设定条件下的模拟地震进行了三次试验,分别测试误差评价时重复反演次数的设定值,误差评定方法对噪声的反应情况,权重优化的效果。第一次试验研究误差评价时反演的重复次数对最终结果的影响,用以为该方法在应用时选择合适的反演次数。试验分别尝试了重复20次,40次,60次,80次和100次,结果发现该方案对反演次数不是特别敏感,结果基本一致且可信。为保证样本量充足,选定100次为默认反演次数。第二次试验测试结果误差对数据噪声的反应是否合理,对理论事件加噪时分别加了低,中,高,超高强度噪声,反演结果确实体现了误差由低到高的趋势,而且理论真值均在误差范围内,表明了本文误差估计的真实性和稳定性。第三次试验则分别设定了单独信噪比加权,单独振幅调节加权,信噪比和振幅调节联合定权三组对照组。三组对照组反演结果均在误差范围内包含真值,但是综合看来联合定权的平均误差是最小的,体现了联合定权的优越性。}

\label{ID_1381199343}\subsubsection{为了体现本文方法的实际应用性,以2013年4月20日的芦山地震为例,分别采取单独振幅调节加权、单独信噪比加权以及振幅调节和信噪比调节联合加权的策略进行三次反演,并利用本文的误差估计方法对三次反演的结果进行评价。通过对结果的合理性及平均误差两方面进行讨论,以真实案例证实本文联合加权策略反演结果最优,对应的震源机制解为(走向200°,倾角38°,滑动角89°),震源深度18km,与其它研究者的研究成果有很好的一致性,且与震源区的应力及地质构造情况均相互吻合。}

\label{ID_64212303}\section{原理分析}

\label{ID_424266573}\subsection{反演方法}

\label{ID_1903486780}\subsubsection{}

\label{ID_1164082214}\subsubsection{其中 $s(t)$为震源时间函数,$Mkj$为地震矩张量。由于$Mkj$最多只有6个相异参数,将其依次记为 $Mi$,利用 $Mkj$的对称性,同时省略 $(x,t)$以及位移分量指标$n$,可将(1)式改写为6项求和的公式(2):}

\label{ID_1121124158}\subsubsection{ 其中$Gi$为格林函数,在层状速度模型的波场理论\citep{Haskell1964,Herrmann1979,Wang1980}基础上, 借助波数积分法\citep{Zhu2002}可计算出任意一维层状速度模型的格林函数。将格林函数与观测波形代入(2)式,通过反演计算即可求得最佳矩张量,实际反演时考虑到震源机制解的全空间(走向0°-360°,倾角0°-90°,滑动角-180°-180°)较小,通常采用全空间格点搜索法。即将全空间划分为一定精度间隔的格点,将每个格点震源机制转换为等价矩张量\citep{Jost1989}并代入(2)式求出相应的理论波形,并计算该理论波形与实际波形的拟合度,逐步计算每一格点对应的拟合度便可求得拟合度最高的全局最优解。}

\label{ID_1321964956}\subsubsection{CAP与CPS方法计算波形的拟合度时使用了不同的目标函数,CAP方法先将理论波形与实测波形通过互相关运算进行时差调整,再计算调整后的波形间残差范数(L1或L2)\citep{Zhao1994},该残差范数定为为最终的目标函数,CPS方法则直接将理论与实测波形的互相关函数作为目标函数。为了单独分析权重因素对反演的影响,对比CPS与CAP不同定权的差异,后文反演试验及讨论均基于CPS的目标函数方案。CPS方法中的互相关目标函数Fit如公式(3)定义:}

\label{ID_143427010}\subsubsection{在计算机中进行数值计算时使用公式(3)所示离散形式:}

\label{ID_247138940}\subsubsection{其中}

\label{ID_457268259}\subsubsection{且}

\label{ID_31082694}\subsubsection{}

\label{ID_412480889}\subsection{定权优化}

\label{ID_1192414214}\subsection{误差评定}

\label{ID_1164850097}\section{理论实验}

\label{ID_1307536851}\subsection{实验设定说明}

\label{ID_146378205}\subsection{误差评定检验}

\label{ID_1408987114}\subsubsection{不同噪声强度测试}

\label{ID_74589139}\subsubsection{不同反演次数测试(选做)}

\label{ID_849140981}\subsection{权重优化检验}

\label{ID_979847281}\subsubsection{不同加权反演,通过误差评价权重优化}

\label{ID_89289087}\section{实例应用}

\label{ID_1189436416}\subsection{案例选取}

\label{ID_887518159}\subsubsection{芦山地震背景说明}

\label{ID_1205767782}\subsubsection{选取该地震的科学优势}

\label{ID_1181782570}\subsection{数据处理}

\label{ID_125212802}\subsection{结果分析与对比}

\label{ID_406104104}\subsection{结论}

\label{ID_1085563922}\section{结论}


\newpage
%\tableofcontents

\end{document}
