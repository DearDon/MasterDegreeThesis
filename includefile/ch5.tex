
%%% Local Variables:
%%% mode: latex
%%% TeX-master: t
%%% End:

\chapter{总结和展望}
以上理论实验和实例应用充分说明了本文所提出的对震源机制误差评价方法的有效性,其准确反映了数据随机噪声的存在对于反演得到的震源机制的影响,明确给出了误差范围。并且还揭示了震源机制各参数间的相关性,对于进一步推测误差的情况起到了指导作用。对于优秀数据质量的情况下的误差仍然不可忽略情况说明了,在实际工作中反演震源机制时评估数据随机噪声的影响的重要性和必要性,肯定了本文工作的意义。

对于权重优化的实验表明了本文联合加权方案在一定程度上对反演结果进行了优化。但是需要谨记,对反演结果有决定作用的是原始数据质量,包括数据信噪比和数据结构。数据信噪比表征了数据信息的真实性,因为随机噪声相当于虚假信号,对结果具有误导作用。数据结构则表征了数据信息总量的约束强度,即使数据没有任何误差,当结构分布很差,数据不足时也无法得到唯一真实解,这相当于欠定反演问题。只有当数据质量达到反演满足的最低要求时,合理地加权才能显示出优化结果的作用。本文联合加权中的信噪比权重W1针对数据信噪比,降低了高噪声数据的影响,并最终提高了代表反演结果可信度的拟合度。而振幅调节权重W2则是从数据结构着手,通过合理平衡不同数据在反演中的影响力,相当于间接“增加”了参与反演的数据数量,增强了反演的约束力度。

对芦山地震的反演展示了本文方法的实用性,对反演结果和他人成果的对比分析表明,余震基本沿主震断层面破裂分布,延展趋势与本文走向匹配;芦山地震所处的南段区域,其山前断裂带明显受到了冲断运动的影响,发生了较为强烈的冲断和摺皱变形,为震源所处的盲逆断层孕震提供了有利条件,与本文所得到的逆冲型断裂发震的运动背景一致;震源区钻井实测资料,快剪切波应力计算资料结果相互吻合,均与本文震源机制的滑动角、倾角所暗示的应力情况表现出一致性,表明芦山地震主要为区域NWW向水平应力长年积累的一次应力释放。

本文最核心的工作是对震源机制反演过程中误差评定给出了一个可行方案,但是仍然有许多值得改进的地方。首先,误差估计仅基于对随机噪声的分析,然而实际情况中,反演中除了数据随机噪声,还有参考模型误差等系统性误差。在本文工作中由于模型偏差等系统性偏差的研究欠缺,便直接忽略了系统误差,事实上该误差的影响可能比随机误差更大,值得在之后的研究中关注。其次,对误差统计分析的关键在于制造解的随机集合,这要求进行大量的重复反演,降低了反演效率,不适合应用于对实时性要求较高的自动化系统中,相对更偏向对速度要求相对宽松的精化研究中。
