
\chapter{总结和展望}
\section{主要创新点}
本文最主要的创新性工作是提出了一种评估震源机制误差信息的方法。鉴于目前国内外利用地震波反演震源机制时,普遍缺少误差评定的现状,并考虑到误差评定对科学研究的重要意义,针对利用地震波拟合反演震源机制中应用较广泛的格点搜索法程序——CPS和CAP,探讨了给予反演误差的思路。首先根据误差信息的相关理论,从其定义出发严谨推导,并得到了相应的理论基础。进而大胆提出了适用于震源机制反演的误差评价方法,评估其可行性并快速实现了方法所需程序。其后慎重考虑了执行中的可能变数,通过理论实验严格检验了该方法的成效。
相比\zhcitet{郑建常}{zhenjianchang2015}随机重采样的误差统计方法,本文方法没有改变原始数据分布,并且保留了所有可用观测数据,因此理论上对观测数据总量要求相对较宽松,适用范围更广。

另一项主要工作是基于CPS和CAP的加权方案,通过权重精化和联合进行了优化。针对CPS和CAP的加权计算方法,比较之下发现其计算的权重数值相对大小存在的矛盾点。随后,考虑了各自加权的主要原因,并深入探讨各方案加权的本质以及对反演的影响,从理论上分析了权重联合的可能性以及预期优化效果。之后对联合加权和单独加权方案进行对比检验,分别从理论实验和实例反演两方面进行了对照组反演,结合结果评价了联合定权的优势。

\section{工作总结}
理论实验和实例应用中对误差的估计和讨论,充分说明了本文所提出针对震源机制误差评价方法的有效性,其准确反映了数据随机噪声的存在对于反演得到的震源机制的影响,明确给出了误差范围。并且还揭示了震源机制各参数间的相关性,对于进一步推测误差的情况起到了一定指导作用。在理论实验中,对于高信噪比数据反演参照组,发现即使在数据噪声含量比较低的情况下,误差仍然不可忽略,证明了在实际工作中反演震源机制时评估数据随机噪声影响的重要性和必要性,肯定了本文工作的意义。
事实上,由于本文误差估计方法基于误差定义的理论基础,因此不仅限于格点搜索反演,也不仅针对震源机制反演,对所有已知数据噪声信息,但未能评价结果误差的反演问题均是一种可行方案。

权重优化的对比实验表明了本文联合加权方案在一定程度上对反演结果进行了优化。但是需要注意,对反演结果有根本决定作用的是原始观测数据质量,包括数据信噪比和数据结构。数据信噪比表征了数据信息的真实程度,因为随机噪声相当于虚假信号,对结果具有干扰作用。数据结构则表征了数据信息总量的约束强度,即使数据没有任何误差,当结构分布很差,数据不足时也无法得到唯一真实解,这相当于欠定反演问题。只有当数据质量达到反演满足的最低要求时,合理地加权才能显示出优化结果的作用。本文联合加权中的信噪比权重W1针对数据信噪比,降低了低信噪比数据的影响,并最终提高了代表反演结果稳定性的拟合度。而振幅调节权重W2则是从数据结构着手,通过合理平衡不同数据在反演中的影响力,相当于间接“增加”了参与反演的数据数量,增强了反演的约束力度,使结果更可靠。

对芦山地震的反演展示了本文误差估计方法的实用性以及权重优化在实际工作中的效果,对本文震源机制(走向$211\degree\pm5\degree$,倾角$41\degree\pm1\degree$,滑动角$94\degree\pm2\degree$)和他人成果的对比分析表明,余震基本沿主震断层面破裂分布,延展趋势与本文走向匹配;芦山地震所处的南段区域,其山前断裂带明显受到了冲断运动的影响,发生了较为强烈的冲断和摺皱变形,为震源所处的盲逆断层孕震提供了有利条件,与本文所得到的逆冲型断裂发震的运动背景一致;震源区钻井实测资料,快剪切波应力计算资料结果相互吻合,均与本文震源机制的滑动角、倾角所暗示的应力情况表现出一致性,表明芦山地震主要为区域NWW向水平应力长年积累的一次应力释放。

通过多次实验结果对比,以及相关理论公式,推测震源机制各参数的误差绝对大小不仅与原始观测数据噪声相关,还和具体的震源机制类型有密切联系。相比之下,各参数误差的相对大小、以及各参数间的两两相关性则主要和地震的震源机制相关,观测数据的噪声对其影响很小。

\section{展望}
本文最核心的工作是针对震源机制反演过程中误差评定缺失问题,给出了一个解决方案——通过模拟噪声和数据进行重复反演,并利用统计手段估计得到误差信息。虽然该方法经过推导、理论实验均证明可靠有效,但是仍然有许多不足及有待改进的地方。

首先,本文明确指出所研究的误差信息仅包括来源于观测数据随机噪声的部分,这并不是表示其它诸如地下速度结构偏差导致的系统性误差在震源研究中不重要。恰巧相反,本文的相关实验暗示了系统性误差的影响不权并非微不足道,而且可能比随机噪声的影响还大。因为基于本文的误差估计,发现在理论实验和真实反演案例中,即使原始数据的噪声强度相近,理论实验的数据拟合度仍显著高于实际地震反演的拟合度。这说明其中还有除数据噪声以外不可忽视的干扰,合理推断应该是来源于真实地下结构与参考模型的差异、以及对真实地震过程进行模型简化后的影响。

从理论上分析,本文的误差估计方法基于数据噪声的随机分布,利用噪声期望与分布的特点,通过重复试验统计分析随机噪声的影响。然而系统性误差不具备这样的特性,无法通过多次重复反演进行消除或评估其影响。因此,从原理上讲,本方法不适用于对系统性误差的研究。如果强行尝试,一种考虑方案是利用随机噪声的方式对待系统性误差,将其随机化,如将参考模型像数据噪声一样给出一定的误差范围,并在反演中考虑其可能偏差的影响。但是每次模型的不同偏差必然需要重新计算格林函数库,格林函数库计算是震源机制反演中最耗时的步骤,这在重复的大量反演中将带来不可接受的计算量,因此不具备实际可行性。因此对系统误差的考虑,仍旧需要进一步研究,寻找其它解决方案。

介于以上原因,在本文工作中由于模型偏差等系统性偏差的研究欠缺,便直接忽略了系统误差,事实上该误差的影响有可能比随机误差更大,值得在之后的研究中关注。

其次,对误差统计分析的关键在于制造解的随机集合,尽可能使误差估计接近真实情况。从概率统计角度考虑,这要求进行大量的重复反演,才能保证误差范围具有相对较高的可信度。但是大量的重复反演,直接倍增了总的计算时间,降低了反演效率。因此本文的误差估计方法不适合应用于对实时性要求较高的自动化系统中,而是相对更适合于对速度要求相对宽松的后期精化研究中。

考虑到上述本文误差评价方法的两点不足,除了进一步优化本文方法,另一种可选方案是直接从其它思路考虑。如\citet{Duputel2012}关注了震源机制反演过程中普遍缺少误差评价的问题,意识到误差估计的重要性,并就此较为系统地讨论了震源机制反演中的各种误差。不过他的研究是基于另一种较新颖的线性化的震源机制反演方法,解决方案对于国内应用更普遍的CAP类搜索方法并不适用。不过以后更多的反演方法可能被广泛接受,因此更完备的误差评价方法也可能受到更多关注。因此从根本的反演算法角度考虑,解决误差评价问题也值得将来进一步思考。
