% !Mode:: "TeX:UTF-8"

%%% 说明: 此部分需要自己填写的内容:  (1) 中文摘要及关键词 (2) 英文摘要及关键词

%%%%%%%%%%%%%%%%%%%%%%%
%%% ------------ 中文摘要 ---------------%%%
%%%%%%%%%%%%%%%%%%%%%%%
\begin{cnabstract}
近年来,利用波形拟合反演震源机制的方法已经越来越普遍,在震源研究中获得了广泛应用。由于震源机制反演问题具有解空间较小,反演公式复杂等特点,一个比较合适的反演算法是格点搜索反演。在实际应用波形拟合研究震源机制的工作中,也证实了格点搜索反演在该问题的适用性。格点搜索算法是指将解空间按一定精度划分为大量网格,并将一个网格范围内无数连续的解当成同一个解,因而实现了解空间的数值离散化。解算时依次遍历所有格点,试探性利用格点值计算相应观测量的理论值,并比较理论值与观测值匹配度,求最优解的过程便是寻求具有最高匹配度的格点值。

格点搜索算法仅需要计算待求模型到观测量的正演公式,因而有效避免了推导相对复杂的震源机制与观测波形间反算公式的过程。然而,在获得便利的同时,也伴随着一个很大缺陷——无法直接评价误差。正是由于格点搜索不需要求算观测数据到估计解的反推公式,因而也无法得到误差传播矩阵,导致不能在搜索到最优解的同时获得对解的误差评价。误差评价的重要性不言而喻,但在实际工作中可以发现绝大部分震源机制的反演研究中均没有提及对震源机制的误差评估。为了获得一定的误差信息,本文基于概率统计原理提出了一种方法。该方法通过利用观测数据的噪声信息重新随机生成噪声,并利用生成的噪声模拟大量的“观测”数据样本,然后对大量模拟数据样本进行多次独立反演,便得到了误差范围内的解集。通过对该解集进行统计分析,不仅可以得到震源机制各参数的误差大小信息,还能得到参数间的相关性信息。为了检验方法的有效性,文中设计了相关实验考查其误差范围的准确性。多次实验均发现本文所提方法较准确地估计了波形数据的随机噪声“传播”给震源机制的误差。

由于震源机制的重要性,日常研究中经常需要进行震源机制反演。为避免重复工作,一些研究者将基于自己反演方案和算法完成的反演程序进行公开,供其它人使用,于是便有了各种开源的反演程序。在这些开源程序中,CAP(Cut And Paste)和CPS(Computer Programs in Seismology)两个程序都是受到广泛应用,且较为成熟的利用格点搜索算法反演震源机制的程序。两个程序分别体现了其作者的在反演中的观点,在CAP程序的相关文献中,详细介绍了反演时所使用的加权基于不同波形间振幅大小的差异,而从CPS的源码中发现其加权主要考虑到不同波形数据信噪比优劣的不同。由于二者的权重均通过震中距估计,进一步分析发现其信赖震中距的权重数值相对大小相互冲突。为了解决该矛盾并吸收两种加权方案中的有益观点,本文提出了综合考虑信噪比和振幅调节两方面的联合加权方案。此外,通过实例计算分析发现震中距难以简单描述信噪比或振幅的真实情况,并因此尝试了用各波形数据信息直接计算信噪比或振幅,以精化权重的数值。经理论实验和实例应用检验,联合定权确实能在一定程度上优化反演结果的稳定性和可靠性。

本文第一章简要介绍了震源机制研究的背景,发展历程和现状。其中着重强调了当前利用波形拟合反演震源机制方法中常见的误差缺失问题,并探讨了CAP和CPS中出现的加权差异和可优化性。之后宏观计划了本文的工作目标和实现方法。第二章对波形拟合反演震源机制的相关理论以及格点搜索算法进行了详尽地推导,并介绍了本文加权优化和误差评价工作的理论基础和具体方案。第三章设计了一系列理论实验,通过实际计算来证明本文所提方法和观点的正确性。第四章则是以2013年芦山地震为例,展示了将本文方法进行真实应用的效果,并对反演结果进行了大量分析,确保结果可靠性的同时对芦山地震给出了相关讨论。第五章是对全文工作和相关思想的总结,并指出了本文方法的不足和改进方向,期望对之后的相关研究起到一点参考作用。

\end{cnabstract}
\vspace{1em}\par\vfill

%%%--------- 关键词 -------- %%%
\cnkeywords{波形反演, 震源机制, 格点搜索, 误差评价}

%%%%%%%%%%%%%%%%%%%%%%%


%%%%%%%%%%%%%%%%%%%%%%%
%%% ------------ 英文摘要 ---------------%%%
%%%%%%%%%%%%%%%%%%%%%%%

\begin{enabstract}
Recently, It's getting more and more common to inverse waveform for source mechanism. Grid search technique turn out to be a good fit for this inversion problem; as the possible source mechanisms are limited and to deduce the inversion equation seems quite difficult. This technique is approved to be effective on application to waveform inversion by experience. Grid search technique is a way to get the best solution from possible solutions: Firstly, we divide the solution space into grids with specified step-length for treating a unit grid with this step-length as a single point, by which we have discretized the solution space. Secondly, we iterate on every single grid to test how it fit to the inversion problem based on a chosen evaluation standard(least squared criterion, etc.). Finally, after iteration, we certainly get a best solution with the highest fit level among the solution space.

By application of grid search technique, we avoid the process of deducing equation to the form of solution; however, it comes with a inherent disadvantage that error estimation isn't provided directly any more. Due to this disadvantage, most of research about getting mechanism by inversion of waveform just ignore the error estimation, although it's so essential to evaluate your solution from observed data; as observed data always comes along with unpredictable noise. To fix this problem,we come up with a new method based on probability theory: Firstly, we modeling the noise and learn how to produce similar noises as many as we need. Secondly, by adding up the new noises and original observed data, we synthesize the simulated data with reasonable noise. Thirdly, we inverse the synthesis dependantly to get a possible mechanism. Then by repeating synthesis and inversion, we get enough mount of possible mechanisms. By analysis of this solution set, we definitely get the error estimation of mechanism. In the end, to testify the validity of the error estimation method provided by this thesis, we run some tests and analyze the result carefully.

Source mechanism is a basic model for other research; so researchers need to get them from time to time. To avoid repeating work, some brilliant programs are provided by their owner to everyone freely, so that one can focus more on his own research. CAP(Cut And Paste) and CPS(Computer Programs in Seismology) turn out to be two very popular programs Among all the open source software for getting mechanism. They both use grid search technique to fulfill the task, while there are some difference in weighting. CAP cares about the influence from amplitude difference in waveforms and apply higher weighting to data with lower amplitude; CPS focus on the viewpoint that data with better SNR(Signal-to-Noise Ratio) should have priority. Furthermore, as CAP and CPS both calculate weighting by source-station distance, the outcome of relative weighting seems contradiction from the two programs. By analysis, we realize the two different weighting method are both reasonable. To fix the contradiction and take advantage of each, we manage to unite them in a way. Besides, we find the amplitude and SNR evaluated from source-station distance are rough and we refine them by estimation from the very waveform information instead of source-station distance. In the end, we certify the improvement of united weighting by some experiments.

In the first chapter we introduce the background of source mechanism, the history of mechanism research and it's current state. Especially the lack of error estimation and the possibility to improve weighting by combination of CAP and CPS are discussed. In the end, the purpose and main work in this thesis is mentioned briefly. In chapter two, we present the deduction theorem of getting mechanism by inversing waveform, grid search technique and the method developed in this thesis to estimate error and improve weighting. In the third chapter, a series of experiments are designed to test the new method. Then in the forth chapter, we formally apply the whole theory to a real earthquake, which happened in 20th April, 2013, in Lushan county. After inversion, a discussion is given to verify the result. Finally, in the last chapter, we give a conclusion of our works and some experiences.

\end{enabstract}
\vspace{1em}\par\vfill

%%%------ 英文关键词 ------- %%%
\enkeywords{waveform inversion, source mechanism, grid search technique, error estimation}
